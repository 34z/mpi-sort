\documentclass{zjureport-zh}

\major{计算机科学与技术}
\course{并行计算与多核编程}
\name{陈卓}
\expname{MPI 并行排序}
\stuid{3170101214}
\instructor{楼学庆}
\college{计算机科学与技术学院}
\title{并行计算与多核编程}


\begin{document}

\makecover

\section{算法}
\par 本次实验使用 MPI 实现了三种并行排序算法。

\subsection{并行归并排序}
\par 本实验并行归并排序的想法十分平凡,即:

\begin{enumerate}[itemindent=2em,label=\roman*)]
	\item $P_0$ 将原数据均分到不同进程 $P_i,\ 0 \leq i < p$;
	\item $P_i$ 各自进行串行快速排序;
	\item 将排序后的数据发送到 $P_0$;
	\item $P_0$ 对这些有序数列进行归并。
\end{enumerate}

\subsection{并行奇偶排序}
\par 奇偶排序是冒泡排序的一个变种。串行奇偶排序的算法步骤如下:
\begin{enumerate}[itemindent=2em,label=\roman*)]
	\item \textbf{偶阶段}:比较并交换 $(a[0], a[1]), (a[2], a[3]), \dots$;
	\item \textbf{奇阶段}:比较并交换 $(a[1], a[2]), (a[3], a[4]), \dots$;
	\item 重复进行 $0, 1, \dots, n$ 阶段。
\end{enumerate}

\par 奇偶排序算法的并行化相对容易,因为它在每个阶段的比较交换是可并行的,但由于不可能提供 $n$ 个处理器进行排序,因此需要将串行算法中比较交换的最小单位更改为一个处理器 $P_i$,其中每个处理器处理 $\frac{n}{p}$ 的数据。并行算法步骤如下:
\begin{enumerate}[itemindent=2em,label=\roman*)]
	\item $P_0$ 将原数据均分到不同进程 $P_i,\ 0 \leq i < p$;
	\item \textbf{偶阶段}:比较并交换 $(P_0, P_1), (P_2, P_3), \dots$;
	\item \textbf{奇阶段}:比较并交换 $(P_1, P_2), (P_3, P_4), \dots$;
	\item \textbf{比较交换}:将 $P_i, P_{i+1}$ 的数据归并,$P_i$/$P_{i+1}$ 取归并后较小/大(大/小)的一半;
	\item 重复进行 $0, 1, \dots, n$ 阶段。
\end{enumerate}

\subsection{PSRS 排序}
\par PSRS 排序在上次实验中已经通过 C++ 线程库实现,这次采用 MPI 实现,算法步骤如下:
\begin{enumerate}[itemindent=2em,label=\roman*)]
	\item $P_0$ 将原数据均分到不同进程 $P_i,\ 0 \leq i < p$;
	\item $P_i$ 将各自数据排序,并选择 $p$ 各\textbf{样本}(第 $0, \frac{n}{p^2}, \dots$ 个元素)发送给 $P_0$;
	\item $P_0$ 对 $p$ 各处理器的样本归并,并选择 $p-1$ 个元素作为\textbf{主元},广播给所有处理器;
	\item $P_i$ 根据 $p-1$ 个主元将各自数据分为 $p$ 段,并将第 $i$ 段发送给 $P_i$;
	\item $P_i$ 对数据段归并,并发送给 $P_0$。
\end{enumerate}


\section{实现}
\par 本实验选用的 MPI 实现是 MPICH,版本 3.3,环境 macOS 10.15.3,C 编译器 Apple clang version 11.0.3。
\par 三种算法通过 MPI 接口实现 所属算法步骤,实现代码详见 \texttt{merge.c}、\texttt{oddeven.c}、\texttt{psrs.c} 文件,或在 \href{https://github.com/zhuo34/mpi-sort}{GitHub} 中查看 。


\section{测试}
\par 本实验测试分别测试比较以上三种并行排序算法与串行快速排序,测试平台 2.3 GHz 双核 Intel Core i5。测试脚本见 \texttt{test.py}。测试结果如下。

\begin{table}[h]
\centering
\caption{并行归并算法}
\vspace{1ex}
\begin{tabular}{|c|c|c|c|c|}
        \hline
        data size & 2 & 4 & 6 & 8 \\
        \hline
        1000 & 0.318 & 0.4144 & 0.2014 & 0.0637 \\
        \hline
        2000 & 0.6028 & 0.5869 & 0.2145 & 0.1704 \\
        \hline
        4000 & 0.9279 & 0.8163 & 0.5669 & 0.2439 \\
        \hline
        8000 & 0.936 & 0.7147 & 0.6749 & 0.1723 \\
        \hline
        10000 & 1.1766 & 0.8111 & 0.5276 & 0.3672 \\
        \hline
        20000 & 1.1211 & 1.4503 & 1.1332 & 0.4771 \\
        \hline
        40000 & 1.0878 & 1.2525 & 1.2845 & 0.6683 \\
        \hline
        80000 & 1.5046 & 1.3068 & 1.3158 & 0.9683 \\
        \hline
        100000 & 1.4502 & 1.2389 & 1.2877 & 0.9347 \\
        \hline
        200000 & 1.5111 & 1.2989 & 1.1516 & 0.8237 \\
        \hline
        400000 & 1.711 & 1.9033 & 1.4051 & 1.2541 \\
        \hline
        800000 & 1.5593 & 1.9909 & 1.7643 & 1.4167 \\
        \hline
        1000000 & 1.6864 & 1.9013 & 1.5149 & 1.495 \\
        \hline
        2000000 & 1.6305 & 1.9528 & 1.8392 & 1.3802 \\
        \hline
        4000000 & 1.6983 & 2.1558 & 1.7956 & 1.0306 \\
        \hline
        8000000 & 1.7761 & 2.2193 & 1.8555 & 1.4662 \\
        \hline
        10000000 & 1.7989 & 2.2553 & 1.8771 & 1.5491 \\
        \hline
\end{tabular}
\end{table}

\begin{table}[h]
\centering
\caption{Odd-even 和 PSRS}
\vspace{1ex}
\subtable[Odd-even] {
	\begin{tabular}{|c|c|c|c|c|}
        \hline
        data size & 2 & 4 & 6 & 8 \\
        \hline
        1000 & 0.628 & 0.2763 & 0.1673 & 0.1116 \\
        \hline
        2000 & 0.58 & 0.7009 & 0.3847 & 0.1702 \\
        \hline
        4000 & 1.1253 & 0.4046 & 0.5251 & 0.1501 \\
        \hline
        8000 & 1.0995 & 0.9898 & 0.6476 & 0.4847 \\
        \hline
        10000 & 1.1438 & 1.6084 & 0.7593 & 0.6653 \\
        \hline
        20000 & 1.6753 & 1.1987 & 1.2355 & 0.6608 \\
        \hline
        40000 & 0.9572 & 1.5913 & 1.0018 & 1.088 \\
        \hline
        80000 & 1.3408 & 1.0079 & 1.9049 & 1.4625 \\
        \hline
        100000 & 1.1573 & 1.2169 & 0.6721 & 1.2137 \\
        \hline
        200000 & 1.4405 & 1.6818 & 1.3073 & 0.9121 \\
        \hline
        400000 & 1.6733 & 2.21 & 1.8663 & 1.4839 \\
        \hline
        800000 & 1.6691 & 2.2688 & 2.3497 & 1.9017 \\
        \hline
        1000000 & 1.7499 & 2.3117 & 2.0255 & 2.0206 \\
        \hline
        2000000 & 1.6462 & 2.5063 & 2.2737 & 2.5497 \\
        \hline
        4000000 & 1.7047 & 2.5242 & 2.5939 & 2.3107 \\
        \hline
        8000000 & 1.8165 & 2.67 & 2.182 & 2.5189 \\
        \hline
        10000000 & 1.8473 & 2.6753 & 2.5478 & 2.4851 \\
        \hline
	\end{tabular}
}
\subtable[PSRS] {
	\begin{tabular}{|c|c|c|c|c|}
        \hline
        data size & 2 & 4 & 6 & 8 \\
        \hline
        1000 & 0.2361 & 0.1523 & 0.0685 & 0.0181 \\
        \hline
        2000 & 0.4711 & 0.1109 & 0.1176 & 0.0139 \\
        \hline
        4000 & 0.7284 & 0.2762 & 0.4648 & 0.156 \\
        \hline
        8000 & 0.8773 & 0.4807 & 0.3726 & 0.4408 \\
        \hline
        10000 & 0.8842 & 0.9156 & 0.455 & 0.3384 \\
        \hline
        20000 & 0.8289 & 0.8443 & 0.6542 & 0.5443 \\
        \hline
        40000 & 1.0907 & 1.0901 & 0.7088 & 0.5826 \\
        \hline
        80000 & 1.3272 & 0.727 & 1.3212 & 0.9155 \\
        \hline
        100000 & 1.2827 & 0.6154 & 0.7847 & 0.879 \\
        \hline
        200000 & 1.5992 & 0.9787 & 1.089 & 0.8136 \\
        \hline
        400000 & 1.5221 & 1.732 & 1.3438 & 1.1163 \\
        \hline
        800000 & 1.4456 & 1.6909 & 1.5921 & 1.3406 \\
        \hline
        1000000 & 1.3622 & 1.9132 & 1.641 & 1.3945 \\
        \hline
        2000000 & 1.44 & 1.9633 & 1.705 & 1.3566 \\
        \hline
        4000000 & 1.4199 & 1.9663 & 1.6449 & 1.3913 \\
        \hline
        8000000 & 1.4382 & 2.0683 & 1.7433 & 1.43 \\
        \hline
        10000000 & 1.6067 & 2.076 & 1.8918 & 1.5807 \\
        \hline
	\end{tabular}
}
\end{table}

\par 表格中,每行代表不同的数据量大小,每列代表 MPI 进程数量,每个元素代表加速比\footnote{取三次平均。}。

\newpage
\section{讨论}
\par 根据测试结果针对不同算法横向比较,我们可以看出并行奇偶排序的加速效果最好,这是因为并行奇偶排序的并行度较高,所有处理器在每个阶段都参与归并运算。而并行归并排序只通过 $P_0$ 归并,并行度较低。同时, PSRS 算法在处理器广播片段时,由于预先不能预测片段的长度,因此在实现上先广播 $p$ 个片段的长度,再广播 $p$ 个片段,造成性能的损失。
\par 由此我们可以提出针对这两种算法的改进方法。并行归并排序中,可以通过 $log$ 递减的方式归并来提高并行度,即第一轮 $(P_0, P_1), (P_2, P_3), \dots$ 两两归并,第二轮 $(P_0, P_2), (P_4, P_6), \dots$ 两两归并,这样可以提高并行度。PSRS 排序中,可以为每个片段申请 $\frac{n}{p}$ 大小的空间,在广播片段时可以不用顾虑片段长度问题。

\par 针对处理器数量比较,以并行奇偶排序算法为例,结果趋近于 $p=4 > p=6 > p=8 > p=2$,4 进程的效率最高是因为测试机器是 4 线程 CPU。其它算法由于 MPI 进程开销暂时无法准确评估,因此顺序略有变化,但是 $p=4$ 效率最高。

\par 纵向比较,$p=2$ 时三种算法的加速比随着数据量的增加,稳定在 $1.6 \sim 1.8$ 之间,有一定的可扩放性。


\end{document}


























